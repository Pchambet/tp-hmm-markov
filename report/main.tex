\documentclass[11pt,a4paper]{article}

\usepackage[utf8]{inputenc}
\usepackage[T1]{fontenc}
\usepackage[french]{babel}
\usepackage{amsmath, amssymb}
\usepackage{graphicx}
\usepackage{hyperref}
\usepackage{booktabs}

\title{TP Chaînes de Markov et Modèles de Markov Cachés}
\author{Pierre Chambet}
\date{\today}

\begin{document}
\maketitle

\section{Introduction}
% Contexte général : chaînes de Markov, HMM, pluie, reconnaissance de mots.

\section{Partie 1 : Modélisation de la pluie par une chaîne de Markov}
% Modèle 2 états, résultats théoriques, simulations, comparaison données réelles.

\section{Partie 2 : HMM à 3 états pour la pluie}
% Modèle nuages, Baum-Welch, comparaison distributions de durées.

\section{Partie 3 : Reconnaissance de mots par HMM gaussiens}
% Features audio, apprentissage HMM, matrices de confusion, F1-score.

\section{Conclusion}
% Ce que ces modèles capturent bien / limites / pistes d'amélioration.

\end{document}